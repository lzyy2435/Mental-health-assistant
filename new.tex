 
% 导言区
\documentclass[a4paper]{ctexart}%book,report,letter
\title{\textbf{\Huge 程序设计综合实验}
	\\ \textit{\huge 需求报告} }
\author{刘志远\quad 2020212174\\
刘威\quad 2020212172\\
沈琪\quad 2020212178}

\date{\today}
\usepackage{xcolor}
\usepackage{longtable}
\usepackage{amsmath}
\usepackage{amssymb}
\allowdisplaybreaks
\usepackage{tikz}
\usetikzlibrary{arrows,shapes,chains}
\usepackage{bm}
% \usepackage{ctex}
%\newcommand{\zhongdian}{\textbf}
\usepackage{graphicx}

% 正文区
\begin{document}
	\maketitle
	\tableofcontents
	\newpage
	\section{任务概述}
	\subsection{目标}
	
	1.在PC端完成针对一个用户的实时情绪分析与疲劳识别,并整合为用户精神健康信息实时输出在GUI上。
	
	2.保存一段时间内用户的精神健康数据,并根据信息提供精神健康的检测结果与改善建议。
	
	3.进阶目标:可以同时识别并记录多人的精神健康情况。
	
	4.进阶目标:添加神态识别模块(如皱眉)以及相关心理微表情的检测分析。
	\subsection{系统(或用户)的特点 }
	\textbf{系统特点:}
		
	可以提供专业的精神健康情况检测结果与改善建议,结合情绪分析与疲劳识别服务,让用户更了解自己的精神状态。
	对于集体单位,可以获得为员工、学生的具体精神状况,以便于关心员工、学生的心理状态与工作、学习效率。
	
	\textbf{用户特点:}
	
	\begin{figure}[h]
		%\small
		\centering
		\includegraphics[width=12cm]{2}
		\caption{用户画像/共感图(以学生用户为例)}
	\end{figure}
	
	产品的用户主要分为一下四类:
	\subitem{1.从事长时间面对屏幕的职业者(文字工作者、程序员等)。此类用户精神状态往往较为低迷,需要在心理问题上得到更多帮助。}
	\subitem{2.长期进行重复性工作的职业者。此类用户人际交往活动不多,渴望得到关心等正反馈来调节心理状态。}
	\subitem{3.心理状态尚未稳定成熟的学生。此类用户的情绪波动较大,使用特点可能不规律而急促。}
	\subitem{4.对自己心理健康状态关注较多的人。此类用户可以使用产品长期规律记录个人心理状况,使用特点为较规律且持续。}
	
	\subsection{假定和约束}
	\textbf{开发期限假定:}
	
	用两周时间进行概要设计,两周时间进行详细设计,六周时间实现调试和集成,在教学周第十六周完成所有内容。
	
	\textbf{开发约束:}
	
	Python的实现:运行效率较慢,倚赖库较多,对平台的适应性较差。
	
	识别检测模块:可能出现检测模型过大,检测时间较长,从而导致识别帧率降低,精度下降,用户体验下降。
	
	数据库的结构及最终实现方式需要在编写过程中进行设计。
	\newpage
	\section{需求规定}
	\subsection{软件功能说明}
	\begin{table}[h]
		\centering
		\addtolength{\leftskip} {-1.5cm}
		\begin{tabular}{|c |c |c |c|}
			\hline
			功能要求&输入量&处理&输出\\\hline
			情绪识别&从用户获得的&通过训练好的&蕴含情绪信息\\
			&实时视频流&CNN网络进行分类&的列表(float list)\\\hline
			倦态检测&从用户获得的&通过68个人脸特征点进行分析&疲劳程度(float)\\
			&实时视频流&计算出哈欠及眼动信息&\\\hline
			汇总精神信息&蕴含精神状态信息&分类汇总&含有精神状态和时间\\
			&的列表(float list)&存储到数据库中&的json文件\\\hline
			精神信息&含有精神状态和时间&进行数据分析&用户一段时间的精神\\
			数据分析&的json文件&&状态信息图表与相关建议\\
			\hline
			*人脸识别&从用户获得的&根据68个人脸特征点	&——\\
			&实时视频流&提取的特征向量&\\\hline
		\end{tabular}
	\caption{软件功能说明表(*为进阶任务)}
	
	\end{table}

   产品容量:一次仅接受一个人脸输入,并且系统一次仅在一个终端上运行,不存在在一次运行中并行操作的用户。
	\subsection{对功能的一般性规定}
	本处仅列出对开发产品的所有功能(或一部分)的共同要求,如要求界面格式统一,统一的错误声音提示,要求有在线帮助等。
	\subsection{用户界面}
	概要描述功能对应的用户界面风格。
	\subsection{对性能的一般性规定}	
	\subsubsection{精度}  
	说明对该系统的输入、输出数据精度的要求,可能包括传输过程中的精度。
	\subsubsection{时间特性要求} 
	说明对于该系统的时间特性要求。  
	\subsubsection{灵活性}  
	说明对该系统的灵活性的要求,即当需求发生某些变化时,该系统对这些变化的适应能力,如:操作方式上的变化;运行环境的变化;同其他软件的接口的变化;精度和有效时限的变化等。 
	\subsubsection{输入输出要求}
	解释各输入输出数据类型,并逐项说明其媒体、格式、数值范围、精度等。对系统的数据输出及必须标明的控制输出量进行解释并举例。  
	\subsubsection{数据管理能力要求}  
	说明需要管理的文卷和记录的个数、表和文卷的大小规模,要按可预见的增长对数据及其分量的存储作出估算。  
	\subsubsection{故障处理要求}
	列出可能的软件、硬件故障以及对各项性能而言所产生的后果和对故障处理的要求。  
	\subsubsection{其他专门要求}
	如用户对安全保密的要求,包括信息加密、信息认证方面的要求;对使用方便的要求,对可维护性、可补充性、易读性、可靠性、运行环境可转换性的特殊要求等。
	
	\section{运行环境规定}
	\begin{figure}[h]
		%\small
		\centering
		\includegraphics[width=10cm]{1}
		\caption{系统数据流简图}
	\end{figure}
	\subsubsection{GUI(图形界面)}
	该部分主要完成以下任务:
	\subitem{1、给予用户更优质的程序交互体验}
	\subitem{2、实时传输来自用户的视频流信息至核心算法部分}
	\subitem{2、实时输出经核心算法分析后的用户精神健康信息}
	\subitem{3、获取数据库中存储的用户精神健康信息以提供检测结果与改善建议}
	
	
	代码由……完成
	\subsubsection{核心算法}
	该部分主要完成以下任务:
	\subitem{1、获取来自GUI的视频流信息}
	\subitem{2、利用核心算法(情绪分析与疲劳识别)实时分析视频流信息}
	\subitem{3、将实时分析的用户精神健康信息在GUI输出给用户并保存部分信息于数据库}
	
	
	代码由python的PyQt库完成
	
	
	\subsubsection{数据库}
	该部分主要完成以下任务:
	\subitem{1、获取并保存来自核心算法部分的的精神健康信息}
	
	代码由SQLite完成  
	\subsection{设备}  
	列出运行该软件所需要的硬件设备。  
	\subsection{支撑软件}
	列出支持软件,包括要用到的操作系统、编译(或汇编)程序、数据库管理系统、测试支持软件等。
	\subsection{接口}  
	简要说明该软件同其他软件之间的公共接口、数据通信协议等。  
	\subsection{控制}  
	说明控制该产品的运行的方法和控制信号,并说明这些制信号的来源。     
	\section{尚需解决的问题}
	列出在需求分析阶段必须解决但尚未解决的问题。
	
	
\end{document}


